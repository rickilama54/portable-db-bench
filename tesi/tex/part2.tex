\part{Performance Analysis}

\chapter{Performance Analisys Introduction}
\emph{"Every database vendor cites benchmark data that proves its database is the fastest, and each vendor chooses the benchmark test that presents its product in the most favorable light"}\cite{burleson}.

In this chapter we will introduce database performance analisys' problem, talking about how to measure the performance, how to understand who is the fastest and how to choose the best database. This preface is not specific to in-memory databases, but can be generalized for every DBMS.
	
	\section{Impartiality Problem}
The most important problem related to performance analysys, and benchmarking, is the validity, understood as impartiality, of the results. 
		
		\subsection{Benchmark History}
The first benchmarks were developed internally by each vendor to compare their database's performance against the competitors. But when these results were published, they weren't considered reliable, because there was an evident conflict of interest between the vendor and its database \cite{gray}.
	
		\subsubsection{Benchmark Wars}
Also when benchmarks were published by third parties, or even by other competitors, the results were always dicredited by loser vendors, who complained for the numbers, starting often a benchmark war. 

A benchmark war started when a loser of an important and visible benchmark reran it using specific enhancements for that particular test, making them get winning numbers. Then the opponent vendor again reran the test using better enhancements made by a "one-star" guru. And so on to "five-star" gurus.

		\subsubsection{Benchmarketing}
Benchmarketing is a variation of benchmark wars. Due to domain-specific benchmarks there was always a benchmark which rated a particular system the best. Therefore the vendors promoted only benchmarks which highlighted the strenths of their product, trying to make them as a standard. This leaded to aploriferation of confuse benchmarks.

		\vspace{0.5cm}
Althought these phenomenons were drastically reduced by the foundation of the Transaction Processing Performance Council (TPC) in 1988, they are still alive.

		\subsection{Benchmark Differences} \label{categories}
It is now evident how hard is to understand which database is the fastest. Benchmarks cannot be properly used to analyze the performance of several databases and choose simply the best. Every benchmark has a bias, testing some particular aspect of database systems, such as writes, reads, transactions and so on. Beyond, every benchmark operation can be implemented in different ways by the same database: creating a connection for every operation or using a connection pool; use different transaction isolation level; load the whole database in RAM or split it in different hard disk partition; etc. Furthemore the same benchmark, with the same implemention for every database, can show dissimilar results when it runs on different platforms (hardware and software).

All these differences are grouped by three categories:
\begin{enumerate}
	\item Test scenario: reads, writes, etc.
	\item Test implementation: there are different way to implement the same transaction.
	\item Execution platform.
\end{enumerate}
	
		\subsection{The Axiom}
By this reasoning comes out the axiom which will guide the following chapters, and the work behind this thesis:

\label{axiom} \emph{"There is not a slower or a faster database: databases are only slower or faster given a specific set of criteria in a given benchmark"}.

This sentece comes from an article of Bernt Johnsen \cite{Bernt}, who, in response to a benchmark comparing HSQL and Derby, stated that it's easy to make a benchmark, but it is always hard to communicate the meaning of the results. The interpretation depends on too many criteria, so that it's not easy to say which database is the fastest, unless specifying the set of criteria used in the benchmark.
	
		\section{Measures}
While analyzing databases performance it's not possible to measure which database is the best, but we can measure many other parameters, which then can be interpreted in order to choose the database that fit our needs.

When choosing a database, the most important features to evaluate and compare are:

\begin{description}

	\item[Throughput] is the number of transactions per second a database can sustain. This is the most important feature to consider when evaluating a database system. The most representative application scenario to understand the meaning of the throughput is an on-ine transaction processing system. This kind of application requires the database to sustain a certain number of transactions per second, based on the number of users, and not every database can suite the needs of the application itself. Another example is real time applications: they require even higher throughput. Therefore it is crucial to understand if a certain database is suitable to an application in terms of transactions per second.
	
	\item[Latency/responsiveness] is the time the database takes to execute a transaction. It can be measured as the inverse of throughput.
	
	\item[File size] of the database image. While traditional DBMS' store objects (tables), indexes and a transaction log file on the file system, in-memory databases, achieving durability, usually store only a journal file containing all the transaction executed on the database. Altought this may seem a small file, it can become very huge, even more than the database image. This measure is interesting whereas each database use different data structures.
	
	\item[RAM usage] is the quantity of RAM a process uses while running. Talking about in-memory database, this can be another way to measure the size of the database image. In addition, this is a critical value to take in mind: IMDBs only works correctly and efficiently under the hypothesis that the RAM is enough to contain the whole database.
	
	\item[CPU load] becomes more important with the increasing number of services that live togheter in the same server. CPU is a precious resource shared by all the processes in a particular computer. While database systems can be usually deployed on a dedicated server, embedded databases live togheter at least with the application which use the database itself. And often in-memory databases are executed in embedded mode.
	
	\item[Disk I/O] shows the usage of the hard disk, the bottleneck for every traditional databases. Altought pure in-memory databases never access to the disk, when adding durability through a transaction log file, disk I/O is a bottleneck for IMDBs too. Therefore this measure is less important than the others because it is possible to take the hypothesis that every databases have the same disk I/O.
	
	\item[Startup time] is the time the database needs to become operational. 
	
	\item[Shutdown time] is the time the database takes to shut down and kill the process.
	
\end{description}
	
	\section{Choosing a Database}
From the previous sections, it's clear how it is difficult to use benchmarks to prove which database is the fastest. Even whit a fair benchmark, which can be very useful to understand the performance of databases, it is still difficult to choose a database: performance is only one factor to consider when evaluating a database \cite{burleson}. 

Other factors to consider are:
\begin{itemize}
	\item The availability of trained DBAs.
	\item The vendor's technical support.
	\item The cost of ownership.
	\item The hardware on which the database will be deployed.
	\item The operating system which will support the database.
\end{itemize}

In other words, it is very difficult to choose the right database for our needs, and, of course, while evaluating databases and benchmars there is absolutely \emph{no winner}.


\chapter{Test Suite}	
In this chapter we will define the tests used to run a benchmark and to analyze performance. The tests are divided into three categories: base test case, load test case and acid test case.


	\section{Why Define Test Scenarios}
The axiom, previously described in paragraph \ref{axiom} at page \pageref{axiom}, expresses clearly the difficulty to analyze databases' performance and how every result obtained in a given benchmark depends on the specific set of criteria used in the benchmark itself. 

In paragraph \ref{categories} there is also a description of the three major categories in which the criteria fall. The first of these categories is test scenario: different test scenarios may show completely different performance results. It's not possible to avoid this behaviour, but we can define clearly every tests so that we will be aware of the differences between them.

	\section{Base Test Case}
Base test case is a collection of very simple tests, which are configured mostly as a race. This is exactly what the major part of benchmarks does, particularly Poleposition \cite{poleposition}.

Every test can execute different read/write operations on the database and all these operations are inside a loop. The word \emph{transaction}, used extensively in the following paragraphs, refers to an execution of the loop, and therefore all the operations inside it.

The key features of these kind of test are:
\begin{itemize}
	\item A fixed number of transaction before the test stop: so tests are configured as a race where every database must execute a certain number of transaction.
	\item A fixed amount of time before the test stop: as an alternative to a fixed number of transaction, every test run for a specific amount of time, executing the maximum number of transactions per second. 
	\item Different kind of object: tests must be able to execute create/retrieve
	\item single task
\end{itemize}


	
		\subsection{Race Test}
		
		
	\section{Load Test Case}
\emph{"I don't buy the fastest car in the market. I don't even buy the fastest car I can afford. I buy a car I can afford that fits my needs... e.g. drive to the northern Norway with 2 grown-ups, 3 kids, a kayak, sleeping bags, a large tent, glacier climbing equipment, food, clothes and so on. Can't do that with a Lamborghini"}\cite{Bernt}.

		\subsection{Real Time Prepaid System}
			\subsubsection{Domain Object}
			\subsubsection{Check Balance Test}
			\subsubsection{Write New Account Test}
			\subsubsection{Manage Call Test}
	\section{Acid Test Case}
Citazione tpc pdf pagina 46 \cite{TPC-C}.

\chapter{Database Benchmark Softwares Overview}
	\section{Benchmark Requirements}
Desirable attributes ... list of attributes \cite{tpc/sigmoid}.
	
	\section{The Open Source Database Benchmark}
	
	\section{Transaction Processing Performance Council}
\emph{Benchmark results are highly dependent upon workload, specific application requirements, and systems design and implementation. Relative system performance will vary as a result of these and other factors. Therefore, TPC-C should not be used as a substitute for a \bfseries{specific customer application} benchmarking when critical capacity planning and/or product evaluation decisions are contemplated}\cite{TPC-C}.
	
	\section{Apache JMeter}
	\section{Poleposition}
	Poleposition fa schifo! \cite{poleposition}!!!
	
	
\chapter{The New In-Memory Database Benchmark Application}
	\section{Functional View}
	\section{Development View}
	\section{Plug-In Architecture}

\chapter{Results' Analysis}
