\chapter{Abstract}

%In-memory database systems represents the latest step in the adaptation of DBMS to embedded systems.

%   1.  What did you do? 
This document introduces the reader to the argument of in-memory database systems, providing a competitive landscaper between several IMDB's vendors, and, subsequently, analyzing and comparing the performance of different systems to each other.

%   2. Why did you do it? What question were you trying to answer? 
The purpose of this document is to find the in-memory database which provides the best performance for a specific application usage: a real time prepaid system, such as a telephone company.


%   3. How did you do it? State methods.
This objective was achieved with the development of a proper test suite, simulating a real time prepaid system, and consequently of a portable benchmark application, which was able to execute the test suite previously defined. 

%   4. What did you learn? State major results. 
The need to create a specific test suite and the portable benchmark application comes from the fact that database's performance depends on a specific set of criteria used by the benchmark itself, and they are mainly: the test scenarios, the test implementation and the platform used to execute the benchmark.

%   5. Why does it matter? Point out at least one significant implication.
This means that there is no the best database system, but there are only good systems for particular tasks, or application usage. Therefore, whenever we need to choose a database systems for an application that need crucial performance, we always need to create a specific test scenario which simulates the real application's behavior. In fact the best benchmark for a database system can only be done by the application itself.


-2 parti in cui � divisa la tesi

-capitolo 1 introduzione
-capitolo 2 overview
-capitolo 3 performance analysys introduction
-capitolo 4 test suite
-capitolo 5 bench overview
-capitolo 6 new bench
-capitolo 7 result
-capitolo 8 conclusion